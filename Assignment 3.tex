\documentclass[12pt,letterpaper]{article}
\usepackage{graphicx}
\usepackage{caption}
\begin{titlepage}
\title{A REPORT ON REASONS WHY STUDENTS DROP OUT OF SCHOOLS}

\author{JEMBA YUSUF\\	REG NO:14/U/6567/EVE\\	STD NO: 214017353}
\end{titlepage}


\begin{document}

\pagenumbering{gobble}
\maketitle{}
\newpage
\pagenumbering{arabic}
\section{INTRODUCTION}
  Education is a fundamental human right as well as catalyst for economic growth and human development.A dropout is anyone who leaves school, college or university without either completing their course of study or transferring to another educational institution.  
\section{BACKGROUND}
In 2000, the international community promised that all children would be, and stay in school by 2015.
However, in 2013, there are still 57 million children out of school; one in ten is denied his/her right to
education. Half of these 57 million children live in Sub-Sahara Africa. Out of school patterns vary
across and within regions, and it is therefore critical to analyse contextual reasons
for non-enrolment and early school dropout. One of the first steps in reaching outof-school
children is to identify who they are and where they live. The challenges are
great. UNESCO estimates that there are globally some 215 million child labourers and more than 150
million children with a disability, while 39,000 girls below the age of 18 are married off every day.

   
\subsection{Problem Statement}
 Uganda is greatly affected by many students dropping out of school which will hinder the rate of economic growth in future. This will help to find out the major causes of school drop that can be analysed to get solutions to such challenges.
  \subsection{Objectives}
  \subsubsection{Main Objectives}
   The study sought to get an in-depth understanding of the causes of primary school dropout and non enrolment, in order to find a lasting panacea to improve on school retention and completion rates in
   Uganda. 
   
\subsubsection{Specific Objectives}
a) To identify the wide variety of causes of non-enrolment and dropping out of school in different regions and contexts in Uganda in relation to policy, poverty, school governance, school environment, culture, family/community and other dimensions of diversity or disparity.\\
b) To analyze factors resulting in dropping out from school at different grade levels and linked to age, i.e. underage for grade, appropriate age for grade and overage for grade.\\
c) To analyze and rank the causes of non-enrolment and early school leaving.
d) To identify push and pull factors and document best practices that address and/or reverse nonenrolment and dropping out.\\
e) To advise on strategies and make recommendations which address the identified challenges for improved future programme interventions.\\ 
 \subsection{Scope}
 This research specifically focuses on both primary and secondary students who drop out from school. Information is collected from any person who is willing to give his view and how he thinks about the children dropping out of school.
 \subsection{Significance}
   The research will help identify the wide variety of causes of non-enrolment and dropping out of school in different regions and contexts in Uganda in relation to policy, poverty, school governance, school environment, culture, family/community and other dimensions of diversity or disparity.
 \subsection{ Methodology}
  
 \begin{figure}[h]
 	\includegraphics[width=0.8\textwidth]{Screenshot1}
 \end{figure}
 \begin{figure}[h]	
	\includegraphics[width=0.8\textwidth]{Screenshot2}
\end{figure}
\begin{figure}[h]
	\includegraphics[width=0.8\textwidth]{Screenshot0}
\end{figure}
 \begin{figure}[h]
	\includegraphics[width=0.8\textwidth]{Screenshot3}
\end{figure}
  The source of the data is to be from various people who are willing to give their view on school dropout, review of literature like use of online journals.
  I collected data using interview with help of ODK COLLECT software and this are one of the screen shots i took.
  
 
\subsection{References}
[1]” unicefr”,21 05 2017
[online].Avilable:https://www.unicef.org/uganda/OUTOFSCHOOLCHILDRENSTUDYREPORTFINALREPORT2014.pdf//






\end{document}